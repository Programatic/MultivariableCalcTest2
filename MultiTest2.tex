\title{Multivariable Calc 2}
\author{Ford Smith}
\date{\today}

\documentclass[12pt]{article}
\usepackage{amsmath}
\usepackage{tikz}
\def\checkmark{\tikz\fill[scale=0.4](0,.35) -- (.25,0) -- (1,.7) -- (.25,.15) -- cycle;} 

\begin{document}
\maketitle \begin{enumerate}

\item By saying $\nabla f = \lambda \nabla g$, we are saying that $\nabla g$ is a constant multiple of $\nabla f$. Since they are both vectors, they can have a cross product. This cross product then must become 0 because they are constant multiples.
$$f(x,y)=xy \quad \nabla f = <y,x>$$
$$g(x,y)=x^2+y^2-9 \quad \nabla g = <2x,2y>$$
$$y=\lambda 2x \quad x=\lambda 2 y \rightarrow \frac{y}{2x} = \frac{x}{2y}$$
$$x=y \quad 2x^2=9 \rightarrow x = y = \pm \frac{3}{\sqrt{2}}$$

\item \begin{enumerate}
\item 
$$T(x, y, z) = x^2 +2y^2 - 3z + 1$$
$$\Delta T  = <2x, 4y, -3>$$
$$<\frac{2}{\sqrt{13}}, 0, -\frac{3}{\sqrt{13}}>$$
\item $$T(3, 2, 1) = 15 \quad 17 = x^2 + y^2$$
$$T(\sqrt{15},1, 1) = 15$$
$$<\frac{3-\sqrt{15}}{\sqrt{1+(3-\sqrt{15})^2}}, \frac{1}{\sqrt{1+(3-\sqrt{15})^2}}, 0>$$
\end{enumerate}

\item $$f(x, y) = e^{2x-y-2}+y+\sin(x-1) \quad x(t) = \cos (5t), y(t) = \sin (5t)$$
$$\frac{df}{dt} = \frac{\partial f}{\partial x} \frac{dx}{dt} + \frac{\partial f}{\partial y} \frac{dy}{dt}$$
$$\frac{\partial f}{\partial x} = 2e^{2x-y-2}+\cos(x-1) \quad \frac{dx}{dt} = -5\sin(t)$$
$$\frac{\partial f}{\partial y} = -e^{2x-y-2}+1 \quad \frac{dy}{dt} = 5\cos(t)$$
$$\left(2e^{2x-y-2}+\cos(x-1)\right)\cdot-5\sin(t) + \left(-e^{2x-y-2}+1\right)\cdot5\cos(t) = \frac{df}{dt}$$

\item $$f(x, y) = xy + x + 2y \quad g(x, y) = xy - 4$$
$$\Delta f = <y+1,x+2> \quad \Delta g = <y, x>$$
$$\frac{y+1}{y} = \frac{x+2}{x} \rightarrow x=2y$$
$$2y^2=4 \rightarrow y=\sqrt{2}, x = 2\sqrt{2}$$

\item \begin{enumerate}
\item S
\item V
\item W
\item Y
\item Q
\item P
\item X
\end{enumerate}
\item $$f:x^2+y^2+z^2-9=0 \quad g:z-x^2-y^2+3=0$$
$$\nabla f = <2x,2y,2z> \quad \nabla g = <-2x,-2y,1>$$
$$\textrm{Tangent plane of f and g respectively: } 4(x-2)-2(y+1)+4(z-2)=0$$
$$-4(x-2)+2 (y+1)+z-2=0$$
$$\vec{n_1} = <4,-2,4> \quad \vec{n_2} = <-4, 2,1>$$
$$\cos(\theta) = \frac{\vec{n_1} \cdot \vec{n_2}}{\parallel \vec{n_1} \parallel \cdot \parallel \vec{n_2} \parallel}$$
$$\cos(\theta) = \frac{-16}{6\sqrt{21}}$$
\item \begin{enumerate}
\item $$f\left(x,y,z\right)=x^3+y^3-z^3 \quad f\left(9,10,12\right) = 1$$
$$f_x = 3x^2 \quad f_y = 3y^2 \quad f_z = -3z^2$$
$$L(x,y,z) = f(9,10,12) + f_x(x-9)+f_y(y-10)+f_z(z-12)$$
$$L(x,y,z) = 1 + 243(x-9)+300(y-10)-432(z-12)$$
$$L(9.001,10.02,12.001)=1+243*.001+300*.02-432*.001=6.811$$
$$f(9.001,10.02,12.001)=6.822999$$
\item $$f(x,y)=x\sqrt{y} \quad f(3,169)=3\sqrt{169}$$
$$f_x=\sqrt{y}=\sqrt{169} \quad f_y=\frac{x}{2\sqrt{y}}=\frac{3}{2\sqrt{169}}$$
$$L(x,y) = 3\sqrt{169} + \sqrt{169}(x-3) + \frac{3}{2\sqrt{169}}(y-169)$$
$$L(3.141,163)=40.1407 \quad f(3,169)=40.1016$$
\end{enumerate}

\item $$<\frac{3}{5}, \frac{4}{5}> \cdot \nabla f = 2 \quad <\frac{-4}{5}, \frac{3}{5}> \cdot \nabla f = 2$$
$$\frac{3}{5}f_x+\frac{4}{5}f_y=2 \quad \frac{-4}{5}f_x+\frac{3}{5}f_y=-1$$
$$f_x (0,0) = 2 \quad f_y(0,0)=1$$
$$2x+y+1=L(x,y) \quad L(0.06, .08) = 1.2$$
\item The interpretation of $f_I$ is that as p is kept constant, the rate of change of consumption is dependent on the change in I which is essentially the change in consumption of beef at 3\$/lb as income per year increases. For instance, if C has units of $\frac{lb}{y}$ - which is derived from household income per year over price of beef per pound - then the partial derivative in respect to I where p is kept constant would result in units of $\frac{lb}{\$}$ because of the definition of the partial derivative: $\lim_{h\rightarrow0}(\frac{f(I+h,p) - f(I,p)}{h})$. Plugging in the units shows that it must demonstrate the change in consumption of beef at p per increase of household income per year.
$$f_p = \frac{4.97-5.00}{.5} = -.06$$
\item 
$$f_x = \frac{1}{2} \quad f_y = 1 \quad f_{xx} = -\frac{1}{4} \quad f_{yy}=-1$$
\begin{align*}
Q_f=f(x_0,y_0)+f_x(x_0,y_0)(x-x_0)+f_y(x_0,y_0)(y-y_0)+ \\
\frac{1}{2}f_{xx}(x_0,y_0)(x-x_0)^2+ \\
f_{xy}(x_0,y_0)(x-x_0)(y-y_0)+ \\
\frac{1}{2}f_{yy}(x_0,y_0)(y-y_0)^2
\end{align*}
$$Q_f = 1+\frac{1}{2}(x-1)+y-\frac{1}{8}(x-1)^2-\frac{1}{2}(x-1)y-\frac{1}{2}(y)^2$$
$$Q_f(1,0) = 1 \checkmark$$
\item \begin{enumerate}
\item $$f(x,y) = 2x^2+y^4-4xy$$
$$f_x = 4x-4y = 0 \quad f_y = 4y^3-4x=0$$
$$(-1,-1), (0,0), (1,1)$$
\item 
$$
\begin{bmatrix}
   f_{xx} & f_{yx} \\
   f_{xy} & f_{yy}
\end{bmatrix}
=
\begin{bmatrix}
   4 & -4 \\
   -4 & 12y^2
\end{bmatrix}$$
\item $$\textbf{D} = f_{xx}f_{yy} - f_{xy}^2 = 48y^2-16$$
$$D(-1,-1) > 0, f_{xx} > 4 \rightarrow \textrm{local minimum}$$
$$D(0,0) < 0 \rightarrow \textrm{saddle point}$$
$$D(1,1) > 0, f_{xx} > 4 \rightarrow \textrm{local minimum}$$
\item $$-\nabla f = -<4x-4y, 4y^3-4x> \quad -\nabla f(3,2) = -<4, 20>$$
\end{enumerate}
\item $$f(x,y)=e^{\frac{x}{y}} \quad f(tx,yx) = e^{\frac{tx}{ty}} = e^{\frac{x}{y}} = t^0f(x,y) \quad n = 0$$
$$f_x(x,y) = \frac{e^{\frac{x}{y}}}{y} \quad f_y(x,y)=\frac{-xe^{\frac{x}{y}}}{y^2}$$
$$\frac{xe^{\frac{x}{y}}}{y} - \frac{xe^{\frac{x}{y}}}{y^2} = (0)f(x,y)$$
\end{enumerate} \end{document}